\documentclass[11pt]{paper}
\special{papersize=140mm,210mm}
\usepackage[english]{babel}
\usepackage[left=30mm,right=20mm,top=30mm,bottom=30mm]{geometry}

%%%%%%%%%%% font Caladea for the whole document %%%
\usepackage{caladea}
\usepackage[T1]{fontenc}
%%%%%%%%%%% font Caladea for the whole document %%%

%%%%%%%%%%% бегущая строка %%%%%%%%%%%
\usepackage{fancyhdr} % to customize the footer and header 
\pagestyle{fancy}
\fancyhf{}
\lhead{Appendix for the FONASO Erasmus Mundus Joint Doctorate Programme Scholarship PhD Application, Lemenkova Polina: Research Proposal. June 15, 2010}
%\chead{}
\rfoot{Page \thepage}
\pagenumbering{roman}% arabic
% Set the right side of the footer to be the page number
%\fancyfoot[R]{\thepage}
%%%%%%%%%%% бегущая строка %%%%%%%%%%%

\usepackage[super]{nth}
\usepackage{amsmath}
\usepackage{graphicx}
\setcounter{tocdepth}{3}
\usepackage[dvipsnames,dvipsnames,svgnames,x11names]{xcolor}

%%%%%%%%% hyper ref setup %%%%%%%%%
\usepackage[colorlinks=true]{hyperref}
\hypersetup{pdftitle={Responses of mangrove ecosystems to coastal changes}, 
	pdfauthor={Lemenkova Polina}, 
	pdfsubject={Research Proposal}, 
	pdfcreator={Lemenkova Polina}, 
	pdfproducer={Lemenkova Polina}, 
	colorlinks=true,linkcolor=blue, 
	citecolor=NavyBlue, 
	filecolor=magenta, 
	urlcolor=blue,
	pdfkeywords={Research Proposal, cartography, mapping, data analysis, spatial analysis, comparative analysis, statistical analysis, algorithm, RS, GIS, geography}
	}
%%%%%%%%% hyper ref setup %%%%%%%%%

\title{%
  Responses of Mangrove Ecosystems to Coastal Changes
 \\ \vspace{1em}
  \large Project synopsis for the FONASO Erasmus Mundus Joint Doctorate Programme} 

\author{Lemenkova Polina}

\date{June 15, 2010}

%%%%%%%%%%%%%%%%%%%%%%%%%% END SETUP %%%%%%%%%%%%%%%%%%

\begin{document}

\maketitle

\begin{abstract}
Project synopsis for Erasmus Mundus FONASO Programme. The research aim is to apply methods of GIS-based spatial analysis for modelling, quantitative estimation of the crucial thresholds for losses and changes in the mangrove ecosystems of north-eastern Australia, Queensland, as well as qualitative analysis of the impact of coastal changes on the mangrove sustainable development, and forecasting the possible scenarios in their future dynamics. Spatial analysis includes the classification of the research area into different zones by means of the environmental conditions and dynamics of the coastal areas to investigate the spatial patterns of the species diversity in the mangrove ecosystems. Mangroves, tropical plants growing in tropical climate with roots submerged in sea water, create unique ecosystems in the transitional intertidal zones between marine and terrestrial areas which give them special environmental characteristics. The highly productive mangrove forests, rich in species and genera diversity, have extremely ecological importance for the world ecology. Besides precious ecological function as habitats for tropical wildlife species, perfect protectors against coastal erosion and green belt increasing local carbon fixation, mangroves are important economic sources of food and firewood timber. The research is suggested to be technically based on ArcGIS 9.3 software as a major tool for spatial analysis and cartographic presentation; ENVI software will be used for processing and analyzing geospatial imagery. GIS-based spatial analysis is suggested to detect close to water zones for depicturing mangrove areas, afterwards the raster processing techniques will be applied. The research data will include scenes from Landsat TM and ETM+ for several years in the same periods. The spatial analysis is based on the classification and investigation of the distribution of mangrove trees, which is necessary for the forecasting and prognosis of the future situation of mangrove ecosystems, and analysis of their response towards the environmental changes.\end{abstract}

\section{Institution and Location}
\emph{Erasmus Mundus Secretariat\\
University of Copenhagen\\
Faculty of Life Sciences\\
Bülowsvej 17\\
1870 Frederiksberg C\\
Denmark}.

\section{Research Area}
Australia, Queensland

\section{Problem}
Mangroves, tropical plants growing in tropical climate with roots submerged in sea water, create unique ecosystems in the transitional intertidal zones between marine and terrestrial areas which give them special environmental characteristics (Tomlinson, 1995). The highly productive mangrove forests, rich in species and genera diversity, have extremely ecological importance for the world ecology. Besides precious ecological function as habitats for tropical wildlife species, perfect protectors against coastal erosion and green belt increasing local carbon fixation, mangroves are important economic sources of food and firewood timber (Hogarth, 2007). In tropical areas they can also serve as buffer zones against tsunami and storms. Mangrove ecosystems react rapidly towards changes in hydrogeological conditions and environment; they are especially vulnerable in the areas of high anthropogenic activities. The importance of these vital resources can hardly be overestimated, yet in the late XX century they covered area of no more than 0.4\% of the total forests over the world (Clough, 1998) which trends to decrease yearly. Nowadays, mangroves are almost the most threatened ecosystems in the damage of extinction, due to the human activities, overexploitation, climate and environmental changes, and sea level rise. In Australia, the mangroves are highly dynamic ecosystems with complex patterns and processes and high role as habitat for local benthic, pelagic and demersal species (Bridgewater, 1999). In north- eastern Australia the mangroves of the riverine estuaries are found in especially considerable variety (Bunt, 1999) and high in biodiversity, therefore, the destruction of these forests and their conversion into aquaculture may lead to the irretrievable consequences in the environment of the coastlines of Queensland. The response of this green belt towards the increasing anthropogenic pressure is determined by the extent of the environmental buffer mechanisms able to maintain their sustainable existence and development. The forecasting of the response of mangroves in the coastal zones towards the environmental changes from the external impacts is necessary for the prognosis of the coastal ecosystems dynamics.

\section{Aims and objectives}
The research aim is to apply methods of GIS-based spatial analysis for modelling, quantitative estimation of the crucial thresholds for losses and changes in the mangrove ecosystems of north-eastern Australia, Queensland, as well as qualitative analysis of the impact of coastal changes on the mangrove sustainable development, and forecasting the possible scenarios in their future dynamics. Spatial analysis includes the classification of the research area into different zones by means of the environmental conditions and dynamics of the coastal areas to investigate the spatial patterns of the species diversity in the mangrove ecosystems. Spatial analysis of the dynamics in coastal zones enables to model scenarios in the behaviour and adaptation of the mangrove trees towards the changing environmental conditions in the coastal areas, to estimate the consequences in the development of the coastal zones, and finally to assess the capability of the mangrove ecosystems to resist the negative impacts caused by their buffer location between the marine and terrestrial areas. The dynamics of the changes will be studied using raster processing of the satellite images, and cover research period of approximately 40 years: since early 1970-s up to the latest time.

\section{Technical Approach}
The research is suggested to be technically based on ArcGIS 9.3 software as a major tool for spatial analysis and cartographic presentation; ENVI software will be used for processing and analyzing geospatial imagery. GIS-based spatial analysis is suggested to detect close to water zones for depicturing mangrove areas, afterwards the raster processing techniques will be applied. The research data will include scenes from Landsat TM and ETM+ for several years in the same periods. The spatial analysis is based on the classification and investigation of the distribution of mangrove trees, which is necessary for the forecasting and prognosis of the future situation of mangrove ecosystems, and analysis of their response towards the environmental changes. The differences between the images will be detected using bands combinations, masked and studied for estimation of the changes in the areas. The raster-based mapping includes supervised classification with training sites of mangrove trees (10-15 set areas) using bands for green, red, NIR and combination of NIR/red bands. The main mapping and managing the GIS-project will be supported in ArcGIS architecture through data exporting and conversion. Examination of the satellite imagery provides with vital information of coastal areas of the most recent changes in mangrove forests, as well as mangrove areas in poor conditions or partially damaged. The forecasting of the coastal changes and their affect towards the mangrove ecosystems is based on the quantitative analysis of the physical-ecological response of mangroves to future conditions in the coastal area, and on the comparative estimation of the present and future states of mangrove ecosystems in space and time. The necessity for such forecasts is determined with increased pressure of changing environmental conditions in coastal zones on mangroves trees, and is essential for evaluating the effects of the coastal changes in different environmental scenarios. The proposed environmental modelling is crucial for mangrove conservation in general, and can be applied to other tropical areas (e.g., Amazon, Indonesia). The project is supposed to be based on the application of spatial modelling principles and GIS methodologies to the case study of mangrove ecosystems in Australia, and will quantitatively evaluate the response of mangrove regions to different environmental scenarios.

\section{Expected Results}
The research work is expected to result in series of analytical maps and forecasting of the response of the large mangrove regions to different environmental scenarios, which includes statistical outcomes, such as estimation of the environmentally changed areas, thematic maps and graphs for visualizing the dynamics coastal ecosystems and prognosis maps illustrating results of the different scenarios of the mangrove dynamics under the conditions of coastal changes. The zonal map of the spatial distribution of mangrove forests over the research area is planned to illustrate the variation of the mangrove forests in response towards varying situations of changed environmental conditions which is connected to the capacity of the ecosystems to resist ecological pressure for buffer zones and degree of human activities. The suggested methodology of the analysis of mangroves behaviour in different environmental conditions of coastal zones can be applied towards other tropical areas with mangrove trees.

\section{Keywords}
Mangrove Forests, Environmental Adaptation, Degradation, Coastal Changes, Ecosystem Dynamics, Spatial Analysis, ArcGIS, ENVI, Landsat TM, North-Eastern Australia.
	
\end{document}

%Changing the font size locally (from biggest to smallest):	
%\Huge
%\huge
%\LARGE
%\Large
%\large
%\normalsize (default)
%\small
%\footnotesize
%\scriptsize
%\tiny

\end{document}