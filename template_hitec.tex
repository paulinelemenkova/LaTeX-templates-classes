\documentclass[10pt]{hitec}
%\special{papersize=140mm,210mm}
\usepackage[english]{babel}

%%%%%%%%%%% font Caladea for the whole document %%%
\usepackage{caladea}
\usepackage[T1]{fontenc}
%%%%%%%%%%% font Caladea for the whole document %%%

%%%%%%%%%%% бегущая строка %%%%%%%%%%%
\usepackage{fancyhdr} % to customize the footer and header 
\pagestyle{fancy}
\fancyhf{}
\lhead{Research Proposal: Application for TÜBİTAK Fellowship, Eurasia Institute of Earth Sciences, Istanbul Technical University. Lemenkova Polina: 2011-08-24}
%\chead{}
\rfoot{Page \thepage}
\pagenumbering{roman}% arabic
% Set the right side of the footer to be the page number
%\fancyfoot[R]{\thepage}
%%%%%%%%%%% бегущая строка %%%%%%%%%%%

\usepackage[super]{nth}
\usepackage{amsmath}
\usepackage{graphicx}
\setcounter{tocdepth}{3}
\usepackage[dvipsnames,dvipsnames,svgnames,x11names]{xcolor}

%%%%%%%%% hyper ref setup %%%%%%%%%
\usepackage{hyperref}
\hypersetup{pdftitle={Ecosystem Adaptation to the Climate Change: Anatolian Plateau, Turkey}, 
	pdfauthor={Lemenkova Polina}, 
	pdfsubject={Research Proposal}, 
	pdfcreator={Lemenkova Polina}, 
	pdfproducer={Lemenkova Polina}, 
	colorlinks=false,
%	colorlinks=true,linkcolor=blue, 
	urlbordercolor=green,
	citecolor=NavyBlue, 
%	filecolor=magenta, 
%	urlcolor=blue,
	pdfkeywords={Research Proposal, cartography, mapping, data analysis, spatial analysis, comparative analysis, statistical analysis, algorithm, RS, GIS, geography}
	}
%%%%%%%%% hyper ref setup %%%%%%%%%

\title{%
 Ecosystem Adaptation to the Climate Change: Anatolian Plateau, Turkey
 \\ \vspace{1em}
  \large Appendix for the Application of Lemenkova Polina for TÜBİTAK Research Fellowship} 

\author{Lemenkova Polina}

\date{August 24, 2011}

%%%%%%%%%%%%%%%%%%%%%%%%%% END SETUP %%%%%%%%%%%%%%%%%%

\begin{document}

\maketitle

\begin{abstract}Research Proposal for the TÜBİTAK Scholarship aimed at research stay at the Eurasia Institute of Earth Sciences, Istanbul Technical University. Research focus is GIS based analysis of the land cover changes in the Anatolian Plateau, Turkey. Special objective is assessment of the adaptation of the ecosystems to climate change by means of spatial analysis. Spatial analysis of the thematic layers is supposed to be performed using Quantum GIS and ILWIS GIS. Spatial analysis of the geodata is aimed to perform physical-geographical overview and landscapes of the Anatolian Plateau using techniques of the satellite images classification. Images are taken at different time, classified and compared. Research is aimed to analyze distribution of various land use types over regions of Eastern Turkey. The proposal is scheduled at two months. 
\end{abstract}

\section{Institution and Location}
\emph{Eurasia Institute of Earth Sciences,\\
Istanbul Technical University\\
Istanbul, Turkey}.

\section{Proposed Period}
\nth{1} May 2012 - \nth{30} June 2012.

\section{Research Statement}

\subsection{Research Area}
Anatolian Plateau, Turkey (Eastern part): Southeastern Anatolia, Central Anatolia, Eastern Anatolia. Special research focus includes the most important  following areas: Eastern Taurus Mountains; Pontic Mountains (Kaçkars); East Anatolian Plateau.

\subsection{Research Questions}
\begin{itemize}
	\item Did the landscapes in eastern Turkey changed over the past 20 years ? If “yes”, demonstrate changes on maps (“earlier” and “nowadays”).
	\item What are the dominating land cover types in the landscapes ?
Visualize on map: “Land cover types in eastern Anatolian Plateau, Turkey”.
\end{itemize}

\subsection{Research Approaches}
\begin{enumerate}
	\item Spatial analysis of land cover types in Eastern Anatolian Plateau 
	\item Environmental mapping of mountainous landscapes
	\item A publication of the research results in a journal or a poster for conference presentation 
\end{enumerate}

\section{Materials and Methods}
\subsection{Data}
Data collecting: GIS open source-based geodata will be used for the analysis of the landscape changes in eastern Turkey. These include aerial and satellite imagery covering different time period (time span of 10-20 years).The sources for these geodata include the following types:
\begin{enumerate}
	\item Multi-source satellite imagery: \href{http://www.usgs.gov/pubprod/}{USGS}, \href{http://glcfapp.glcf.umd.edu/}{The Earth Science Data Interface}, \href{http://glovis.usgs.gov/}{GloVis}
	\item open source \href{http://geomorphometry.org/content/data-sets}{Digital Elevation Models} or \href{http://www.gdem.aster.ersdac.or.jp/search.jsp}{ASTER GDEM}
	\item Aerial imagery from the Google Earth server (earth.google.com)
\end{enumerate}

\subsection{Technical Background}
The research will be technically based on following open source GIS software: Quantum GIS, ILWIS GIS for raster processing and geodata modelling. The ArcGIS can be used as well (depending on license availability).

\section{Research Plan}
Detailed research plan is scheduled on two-week periods. 

\subsection{GIS Project}

Data organizing in a GIS project.
\paragraph{01 May – 15 May} The initial part of the research working plan includes:
\begin{itemize}
	\item organizing previously collected geodata on the Institute's computer on a hard disk
	\item data assessment, overview of the available resources
	\item ArcGIS /ILWIS GIS: geo-referencing, organizing geodata
\end{itemize}

\subsection{Spatial Analysis}
\paragraph{15 May – 01 June} 
\begin{enumerate}
	\item Spatial analysis of the thematic layers. Spatial analysis of the geodata is aimed to perform physical-geographical overview and landscapes of the Anatolian Plateau and to analyze distribution of various land use types over regions of Eastern Turkey
	\item Thematic layers in ArcGIS and/or ILWIS GIS: geomorphology of Eastern Anatolia \& Taurus mountains, DEM, landscapes, land cover types	\item Image processing using supervised classification (ILWIS GIS) of land cover types in the Anatolian Plateau and Eastern Taurus mountains
	\item Environmental change detection: overlay of the geodata (aerial photographs, DEMs and satellite images) using cartographic methods, to compare changes in landscape patterns at present and 10-20 years ago (time span 1990 – 2010).
\end{enumerate}

\subsection{Thematic Mapping}
\paragraph{01 June – 15 June} Thematic mapping of the land cover types in Anatolian Plateau
\begin{itemize}
	\item Regional studies of East Anatolian Plateau: land cover types \& landscape patterns
	\item Creating thematic maps detecting landscape dynamics since 1990
	\item Creating thematic maps demonstrating land cover types over the research area.
\end{itemize}

\subsection{Final Report}
\paragraph{15 June – 30 June}: Preparing final report writing paper for publication
\begin{enumerate}
	\item Writing paper for the publication in a journal: “Environmental changes in mountainous landscapes: Anatolian Plateau, Turkey”.
	\item Preparing final report for the TÜBİTAK, Ankara.
\end{enumerate}

\section{Expected Results}
\begin{itemize}
	\item Analysis of the landscape changes in Anatolian Plateau (Eastern part): Southeastern Anatolia, Central \& Eastern Anatolia
	\item Thematic map of the land cover types in Eastern Anatolian Plateau
	\item Report for the TÜBİTAK resuming research results
	\item Paper reporting the research outcome for a journal or a conference poster 
\end{itemize}

	
\end{document}

%Changing the font size locally (from biggest to smallest):	
%\Huge
%\huge
%\LARGE
%\Large
%\large
%\normalsize (default)
%\small
%\footnotesize
%\scriptsize
%\tiny

\end{document}